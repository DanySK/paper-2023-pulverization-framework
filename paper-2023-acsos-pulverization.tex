\documentclass[conference]{IEEEtran}
\IEEEoverridecommandlockouts
% The preceding line is only needed to identify funding in the first footnote. If that is unneeded, please comment it out.
\usepackage{cite}
\usepackage{amsmath,amssymb,amsfonts}
\usepackage{algorithmic}
\usepackage{graphicx}
\usepackage{textcomp}
\usepackage{xcolor}
\def\BibTeX{{\rm B\kern-.05em{\sc i\kern-.025em b}\kern-.08em
    T\kern-.1667em\lower.7ex\hbox{E}\kern-.125emX}}
    
\newcommand{\meta}[1]{{\color{blue}#1}}    
    
\begin{document}

\title{Declarative runtime reconfiguration of pulverised distributed systems
\thanks{Identify applicable funding agency here. If none, delete this.}
}

\author{\IEEEauthorblockN{1\textsuperscript{st} Danilo Pianini}
\IEEEauthorblockA{\textit{Department of Computer Science and Engineering} \\
\textit{Alma Mater Studiorum---Università di Bologna}\\
Cesena, Italy \\
danilo.pianini@unibo.it}
\and
\IEEEauthorblockN{2\textsuperscript{nd} Roberto Casadei}
\IEEEauthorblockA{\textit{Department of Computer Science and Engineering} \\
\textit{Alma Mater Studiorum---Università di Bologna}\\
Cesena, Italy \\
roby.casadei@unibo.it}
\and
\IEEEauthorblockN{3\textsuperscript{rd} Nicolas Farabegoli}
\IEEEauthorblockA{\textit{Department of Computer Science and Engineering} \\
\textit{Alma Mater Studiorum---Università di Bologna}\\
Cesena, Italy \\
email address or ORCID}
\and
\IEEEauthorblockN{4\textsuperscript{th} Mirko Viroli}
\IEEEauthorblockA{\textit{Department of Computer Science and Engineering} \\
\textit{Alma Mater Studiorum---Università di Bologna}\\
Cesena, Italy \\
email address or ORCID}
}

\maketitle

\begin{abstract}
In recent years, we witnessed a radical change in the deployed form of distributed systems.
%
Modern applications are designed to be executed on very diverse devices
and to be deployed on heterogeneous infrastructures,
ranging from cloud servers to mobile and IoT devices.
%
De facto, complex distributed computation is happening across a cloud-edge continuum composed of heterogeneous devices and infrastructures.
%
The availablity of such an infrastructure opens to new possibilities in terms of better resource utilisation and performance,
but also poses new challenges to the application designer,
as the application must be conceived to be able to adapt its deployment to changing conditions.
%
In this paper we present a framework for the development of distributed systems
based on the concept of \emph{pulverisation},
which is meant to neatly separate business logic and deployment concerns,
allowing applications to be defined independent of the infrastructure they will execute upon.
%
The framework is based on a domain-specific language capturing,
in a declarative fashion,
the concepts composing a pulverised application and any potential target network,
also allowing for the specification of reconfiguration policies.
%
The framework, implemented in Kotlin multiplatform and available as open source,
is then evaluated in a small-scale real-world demo and in a city-scale simulated scenario.
\end{abstract}

\begin{IEEEkeywords}
runtime reconfiguration, distributed systems, self-adaptation, self-organisation, pulverisation
\end{IEEEkeywords}

\section{Introduction}\label{sec:introduction}

\begin{itemize}
    \item need for deployment-independent specs
    \item need for runtime reconfiguration (achieve QoS in face of changing conditions --- green computing?)
    \item declarativity over imperative approaches (note: can be compared with the trend in general softeng/build systems)
\end{itemize}

+ related work

\section{A framework for declarative runtime reconfiguration}\label{sec:contribution}

\subsection{Domain model}

\subsection{Domain-specific language}

\subsection{Implementation details}

Also mention that is multiplatform

\section{Evaluation}\label{sec:evaluation}

\subsection{Small scale: crowd alert}

\subsection{Large scale: smart city application}
\cite{PianiniJOS2013}


\section{Related Work}


\meta{
%
%\subsection{Approaches to Deployment Independence}
%\label{s:rw:deployment-independence}
%
%\meta{TODO: add works on deployment independence, e.g., osmotic~\citep{DBLP:journals/computer/VillariFDRJR19}, BIP~\citep{lekidis2015bip-wsn,bastarrica2001optimization-techniques-deployment-components-bip}}
%
We highlight a number of representatieve research efforts on deployment and automatic reconfiguration of systems.
%
%For instance, 
\emph{Osmotic computing}~\cite{DBLP:journals/computer/VillariFDRJR19} is an approach to opportunistic deployment of microservices on the edge-fog-cloud platform.
An osmotic platform aims to reach and maintain an ``osmotic equilibrium'' 
between infrastructural and application requirements
by automatically migrating microservices to deployment locations.
%
However, the approach mainly targets centrally orchestrated systems.
%
Other approaches leverage component-based, architectural descriptions to decouple application logic and deployment.
%
For instance, \emph{DR-BIP (Dynamically Reconfigurable - Behaviour Interaction Priority model)}~\cite{,Ballouli18dr-bip}
and \emph{DReAM (Dynamically Reconfigurable Architectural Modelling)}~\cite{denicola2020dream-dynamic-reconfig-arch-modelling}
use \emph{components} (capturing behaviour), \emph{connectors} (capturing interaction between components' \emph{ports}), 
\emph{maps} (logical topologies),
and \emph{deployments} (associating components to map locations),
overall organised in \emph{motifs} (dynamic architectural configurations),
to model and analyse dynamic architectures.
%

These approaches have some similarity with the approach presented in this paper, but they are arguably more complex and our  
%than the one introduced in this paper,
 %, though also fostering an exogenous, declarative (logics-based) modelling approach.
approach explicitly addresses self-organising CPS.
%
%
%By constrast, our approach explicitly addresses self-organising CPS.
%
%A more in-depth, formal comparison is left as future work.
}

\section{Conclusion and future work}\label{sec:conclusion}

global policies

\bibliographystyle{IEEEtran}
\bibliography{bibliography}

%\vspace{12pt}
%\color{red}
%IEEE conference templates contain guidance text for composing and formatting conference papers. Please ensure that all template text is removed from your conference paper prior to submission to the conference. Failure to remove the template text from your paper may result in your paper not being published.

\end{document}
